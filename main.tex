\documentclass[10pt, letterpaper]{article}

% Packages:
\usepackage[
    ignoreheadfoot, % set margins without considering header and footer
    top=2 cm, % seperation between body and page edge from the top
    bottom=2 cm, % seperation between body and page edge from the bottom
    left=2 cm, % seperation between body and page edge from the left
    right=2 cm, % seperation between body and page edge from the right
    footskip=1.0 cm, % seperation between body and footer
    % showframe % for debugging 
]{geometry} % for adjusting page geometry
\usepackage{titlesec} % for customizing section titles
\usepackage{tabularx} % for making tables with fixed width columns
\usepackage{array} % tabularx requires this
\usepackage[dvipsnames]{xcolor} % for coloring text
\definecolor{primaryColor}{RGB}{0, 0, 0} % define primary color
\usepackage{enumitem} % for customizing lists
\usepackage{fontawesome5} % for using icons
\usepackage{amsmath} % for math
\usepackage[
    pdftitle={Ivan's CV},
    pdfauthor={Ivan Evdokimov},
    pdfcreator={LaTeX with RenderCV},
    colorlinks=true,
    urlcolor=primaryColor
]{hyperref} % for links, metadata and bookmarks
\usepackage[pscoord]{eso-pic} % for floating text on the page
\usepackage{calc} % for calculating lengths
\usepackage{bookmark} % for bookmarks
\usepackage{lastpage} % for getting the total number of pages
\usepackage{changepage} % for one column entries (adjustwidth environment)
\usepackage{paracol} % for two and three column entries
\usepackage{ifthen} % for conditional statements
\usepackage{needspace} % for avoiding page brake right after the section title
\usepackage{iftex} % check if engine is pdflatex, xetex or luatex

% Ensure that generate pdf is machine readable/ATS parsable:
\ifPDFTeX
    \input{glyphtounicode}
    \pdfgentounicode=1
    \usepackage[T1]{fontenc}
    \usepackage[utf8]{inputenc}
    \usepackage{lmodern}
\fi

\usepackage{charter}

% Some settings:
\raggedright
\AtBeginEnvironment{adjustwidth}{\partopsep0pt} % remove space before adjustwidth environment
\pagestyle{empty} % no header or footer
\setcounter{secnumdepth}{0} % no section numbering
\setlength{\parindent}{0pt} % no indentation
\setlength{\topskip}{0pt} % no top skip
\setlength{\columnsep}{0.15cm} % set column seperation
\pagenumbering{gobble} % no page numbering

\titleformat{\section}{\needspace{4\baselineskip}\bfseries\large}{}{0pt}{}[\vspace{1pt}\titlerule]

\titlespacing{\section}{
    % left space:
    -1pt
}{
    % top space:
    0.3 cm
}{
    % bottom space:
    0.2 cm
} % section title spacing

\renewcommand\labelitemi{$\vcenter{\hbox{\small$\bullet$}}$} % custom bullet points
\newenvironment{highlights}{
    \begin{itemize}[
        topsep=0.10 cm,
        parsep=0.10 cm,
        partopsep=0pt,
        itemsep=0pt,
        leftmargin=0 cm + 10pt
    ]
}{
    \end{itemize}
} % new environment for highlights


\newenvironment{highlightsforbulletentries}{
    \begin{itemize}[
        topsep=0.10 cm,
        parsep=0.10 cm,
        partopsep=0pt,
        itemsep=0pt,
        leftmargin=10pt
    ]
}{
    \end{itemize}
} % new environment for highlights for bullet entries

\newenvironment{onecolentry}{
    \begin{adjustwidth}{
        0 cm + 0.00001 cm
    }{
        0 cm + 0.00001 cm
    }
}{
    \end{adjustwidth}
} % new environment for one column entries

\newenvironment{twocolentry}[2][]{
    \onecolentry
    \def\secondColumn{#2}
    \setcolumnwidth{\fill, 4.5 cm}
    \begin{paracol}{2}
}{
    \switchcolumn \raggedleft \secondColumn
    \end{paracol}
    \endonecolentry
} % new environment for two column entries

\newenvironment{threecolentry}[3][]{
    \onecolentry
    \def\thirdColumn{#3}
    \setcolumnwidth{, \fill, 4.5 cm}
    \begin{paracol}{3}
    {\raggedright #2} \switchcolumn
}{
    \switchcolumn \raggedleft \thirdColumn
    \end{paracol}
    \endonecolentry
} % new environment for three column entries

\newenvironment{header}{
    \setlength{\topsep}{0pt}\par\kern\topsep\centering\linespread{1.5}
}{
    \par\kern\topsep
} % new environment for the header

\newcommand{\placelastupdatedtext}{% \placetextbox{<horizontal pos>}{<vertical pos>}{<stuff>}
  \AddToShipoutPictureFG*{% Add <stuff> to current page foreground
    \put(
        \LenToUnit{\paperwidth-2 cm-0 cm+0.05cm},
        \LenToUnit{\paperheight-1.0 cm}
    ){\vtop{{\null}\makebox[0pt][c]{
        \small\color{gray}\textit{Last updated in September 2024}\hspace{\widthof{Last updated in September 2024}}
    }}}%
  }%
}%

% save the original href command in a new command:
\let\hrefWithoutArrow\href

% new command for external links:


\begin{document}
    \newcommand{\AND}{\unskip
        \cleaders\copy\ANDbox\hskip\wd\ANDbox
        \ignorespaces
    }
    \newsavebox\ANDbox
    \sbox\ANDbox{$|$}

    \begin{header}
        \fontsize{25 pt}{25 pt}\selectfont Ivan Evdokimov

        \vspace{5 pt}

        \normalsize
        \mbox{Manchester, UK}%
        \kern 5.0 pt%
        \AND%
        \kern 5.0 pt%
        \mbox{\hrefWithoutArrow{mailto:ivevdm@gmail.com}{ivevdm@gmail.com}}%
        \kern 5.0 pt%
        \AND%
        \kern 5.0 pt%
        \mbox{\hrefWithoutArrow{https://linkedin.com/in/ivan-evdokimov}{linkedin.com/in/ivan-evdokimov}}%
        \kern 5.0 pt%
        \AND%
        \kern 5.0 pt%
        \mbox{\hrefWithoutArrow{https://github.com/ivan020}{github.com/ivan020}}%
    \end{header}

    \vspace{5 pt - 0.3 cm}



    \section{Experience}

        
        \begin{twocolentry}{
            Jan 2023  – Present
        }
            \textbf{AI Engineer}, UK Data Archive -- Colchester, UK\end{twocolentry}

        \vspace{0.10 cm}
        \begin{onecolentry}
            \begin{highlights}
              \item Developed end-to-end ML systems for automated metadata control and classification, deployed via AWS (Lambda, Step Functions, EFS).
              %\item Worked on the development of Model Context Protocol (MCP) applications to automate data retrieval and classification pipelines.
              \item Created a hierarchical ML pipeline utilising tree-based models (scikit-learn) with custom Feed-Forward and LSTM architectures (PyTorch) for NLP-classification tasks.
              \item Fine-Tuned large language models (Ollama and Gemini LLMs) using Low Rank Adaptation (LoRA), later deployed to production as part of metadata validation pipeline.
              \item Developed and maintained programs for metadata processing pipelines in accordance with SOLID principles in Python and C programming languages.
            \end{highlights}
        \end{onecolentry}


        \vspace{0.2 cm}

        \begin{twocolentry}{
            Sep 2022 – Dec 2022
        }
            \textbf{Research Officer and Laboratory Assistant}, University of Essex -- Colchester, UK\end{twocolentry}

        \vspace{0.10 cm}
        \begin{onecolentry}
            \begin{highlights}
            \item Applied statistical and ML regression models to solve theoretical macroeconomic models.
                  %\item Formulated and optimized mathematical problems in theoretical macroeconomic models.
                  %\item Developed a program to simulate macroeconomic processes.
                  \item Assisted in teaching C/C++, Data Structures \& Algorithms, and Introductory Machine Learning modules at postgraduate level.
            \end{highlights}
        \end{onecolentry}
    
        \vspace{0.2 cm}

        \begin{twocolentry}{
            May 2020 – Sep 2020
        }
            \textbf{Analyst Intern}, Beyond Borders Investment Strategies -- Boston, MA, USA\end{twocolentry}

            \vspace{0.10 cm}
            \begin{onecolentry}
                \begin{highlights}
                \item Performed quantitative analysis of exposure of international Exchange Traded Funds (ETFs) to currency and commodity prices fluctuations.
                \item Proposed ideas for mitigation of currency risks via hedging strategies.
                \item Developed a program to extract textual information from the web and pdf sources.
                \end{highlights}
            \end{onecolentry}


    \section{Education}

        \begin{twocolentry}{
            Oct 2021 – May 2025
        }
            \textbf{University of Essex}, PhD in Computational Finance\end{twocolentry}

        \vspace{0.10 cm}
        \begin{onecolentry}
            \begin{highlights}
                % \item Thesis: Innovations to fundamental stock valuations: Estimating future earnings per share and free cash flows using statistical and machine learning methods.
            \item Developed the Transfer Learning-based Bayesian Averaging framework for modeling future fundamental finanical variables.
                \item Led workshops on Linux, Software Development Principles and Dev Tooling.
                \item Led internal seminars on applications of ML techniques to finance.
                \item Facilitated pair programming sessions with fellow PhD-students on projects in GANs, Reinforcement Learning, Genetic Programming, etc.
            \end{highlights}
        \end{onecolentry}

    
        \begin{twocolentry}{
            Oct 2020 – Sep 2021 
        }
            \textbf{University of Essex}, MSc in Financial Econometrics\end{twocolentry}

        \vspace{0.10 cm}
        \begin{onecolentry}
            \begin{highlights}
            \item Built a simulation to model the behaviour of banks, households, and firms under dynamic and negative interest rates in the C programming language.
            \end{highlights}
        \end{onecolentry}

    \section{Publications}

    \begin{onecolentry}
        \begin{highlights}
            \item (In Progress) Evdokimov I., Kampouridis, M., Papastylianou, T., "Deriving Fundamental Stock Value Using Transfer Learning and Earnings-Per-Share".
            \item Evdokimov, I., Kampouridis, M., Papastylianou, T., "Application Of Machine Learning Algorithms to Free Cash Flows Growth Rate Estimation", International Neural Network Society Workshop on Deep Learning Innovations and Applications (INNS DLIA), Procedia Computer Science, Elsevier (2023).
            \item Evdokimov, I., Lungley, D., Rumiancev, A., "Survey Variables Classification with Hierarchical Machine Learning", 15th Annual European DDI User Conference (EDDI 2023).
        \end{highlights}
    \end{onecolentry}

    \section{Technologies}
        
        \begin{onecolentry}
            \textbf{Languages:} Python, C, JavaScript.
        \end{onecolentry}

        \vspace{0.2 cm}

        \begin{onecolentry}
            \textbf{Tools:} NumPy, Scikit-Learn, PyTorch, GCC, Node.
        \end{onecolentry}

        \vspace{0.2 cm}

        \begin{onecolentry}
            \textbf{Development:} Test Driven Development, CI/CD, Git, Agile.
        \end{onecolentry}
    

\end{document}

